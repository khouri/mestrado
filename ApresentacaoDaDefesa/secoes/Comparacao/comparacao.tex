\section{Comparação dos experimentos}

\begin{frame}		
	\begin{block}{Comparação dos experimentos}
		Resultados dos sistemas de recomendação.
		\bgroup
		\begin{table}[!htp]
			\centering
			\tiny
			\begin{tabular}{|l|l|l|l|l|l|l|l|l|} \hline
				\textbf{\(\mathbf{\#}\)} & \textbf{Técnica}&\textbf{S@1}&\textbf{S@5} & \textbf{S@10} & \textbf{S@50} & \textbf{S@100} & \textbf{S@280} & \textbf{MRR} \\ \hline
				
\rowcolor{roxo}		1  & Aleatório				& 0,0037 & 0,0260 & 0,0280 & 0,0300 & 0,0400 & 1,0000 & 0.033 \\ \hline
\rowcolor{roxo}		2  & \emph{Apriori}			& 0,0037 & 0,0385 & 0,0559 & 0,0568 & 0,0570 & 1,0000 & 0,037 \\ \hline
\rowcolor{amarelo}	3  & KNN\(_C\)				& 0,0037 & 0,0685 & 0,0959 & 0,5068 & 1,0000 & 1,0000 & 0,040 \\ \hline
\rowcolor{amarelo}	4  & Rede neural\(_C\)		& 0,0137 & 0,1507 & 0,1781 & 0,8082 & 1,0000 & 1,0000 & 0,089 \\ \hline
\rowcolor{amarelo}	5  & CART\(_C\)				& 0,0274 & 0,1233 & 0,3699 & 0,7671 & 1,0000 & 1,0000 & 0,113 \\ \hline
\rowcolor{amarelo}	6  & Naive Bayes\(_C\)     	& 0,0274 & 0,1507 & 0,3425 & 0,6301 & 1,0000 & 1,0000 & 0,114 \\ \hline
\rowcolor{verde}	7  & Binomial\(_R\) 		& 0,0822 & 0,1918 & 0,2055 & 0,8493 & 1,0000 & 1,0000 & 0,136 \\ \hline
\rowcolor{verde}	8  & Rede neural\(_R\)     	& 0,1096 & 0,2603 & 0,2603 & 0,2603 & 1,0000 & 1,0000 & 0,154 \\ \hline
\rowcolor{verde}	9  & MARS\(_R\)     		& 0,1233 & 0,2055 & 0,2192 & 0,7260 & 1,0000 & 1,0000 & 0,167 \\ \hline
\rowcolor{verde}	10 & SVM\(_R\)     			& 0,1233 & 0,3151 & 0,4932 & 0,8493 & 1,0000 & 1,0000 & 0,238 \\ \hline
\rowcolor{verde}	11 & CART\(_R\)    			& 0,1370 & 0,1370 & 0,2603 & 0,6164 & 1,0000 & 1,0000 & 0,114 \\ \hline
\rowcolor{roxo}		12 & FES           			& 0,1474 & 0,2603 & 0,3699 & 0,8671 & 1,0000 & 1,0000 & 0,196 \\ \hline
\rowcolor{amarelo}	13 & SVM\(_C\)    			& 0,2425 & 0,4658 & 0,4932 & 0,7123 & 1,0000 & 1,0000 & 0,244 \\ \hline
\rowcolor{azul}		14 & SVM composto\(_C\)		& 0,2515 & 0,4458 & 0,5232 & 0,7623 & 1,0000 & 1,0000 & 0,314 \\ \hline
\rowcolor{azul}		15 & Rotation Forest\(_C\)  & 0,2925 & 0,4558 & 0,5432 & 0,7723 & 1,0000 & 1,0000 & 0,324 \\ \hline
\rowcolor{vermelho}	16 & FESO          			& 0,3425 & 0,4658 & 0,5932 & 0,8123 & 1,0000 & 1,0000 & 0,334 \\ \hline
			\end{tabular}
			%\caption{Resultados dos sistemas de recomendação}
			%\label{tb_resultadosExperimentos}
			%\vspace{0.1cm}
			%\source{\varAutorData}
		\end{table}
		\egroup
		
	\end{block}
\end{frame}

%Falar isso, não devo ler
%\begin{frame}		
%	\begin{block}{Comparação}
%		A técnica FESO, proposta nessa dissertação, apresentou um resultado superior às demais. Este resultado ocorre, pois considera o uso de frequência, entrada e saída e informações semânticas sobre as atividades. Em comparação com as demais técnicas seu resultado foi superior para todas as métricas calculadas, exceto \(S@50\) para algumas técnicas. Em relação à técnica FES, seu resultado foi superior. Em particular, parte dessa melhora é justificada pelos casos em que a atividade correta teria frequência zero no conjunto de treinamento, pois ela permite recomendar baseada na ontologia (usando as atividades que contenham a ontologia do novo \emph{workflow}). Além disso, para o caso em que há empate entre duas atividades com o critério de entrada e saída e a frequência a técnica proposta apresenta um fator a mais para ser utilizado como desempate.		
%	\end{block}
%\end{frame}


\begin{frame}
	\begin{block}{Comparação}
		\begin{enumerate}
			\item O aumento de informação melhorou a recomendação.
			\item Regressores foram melhores que classificadores (com exceção do SVM).
			\item Classificadores compostos obtiveram um bom desempenho.
			\item Converter valores contínuos com limiares possibilitou um bom desempenho no caso dos classificadores compostos.
		\end{enumerate}
		
	\end{block}
\end{frame}

