% Template for PLoS
% Version 1.0 January 2009
%
% To compile to pdf, run:
% latex plos.template
% bibtex plos.template
% latex plos.template
% latex plos.template
% dvipdf plos.template

\documentclass[10pt]{article}

% amsmath package, useful for mathematical formulas
\usepackage{amsmath}
% amssymb package, useful for mathematical symbols
\usepackage{amssymb}

% graphicx package, useful for including eps and pdf graphics
% include graphics with the command \includegraphics
\usepackage{graphicx}

% cite package, to clean up citations in the main text. Do not remove.
\usepackage{cite}

\usepackage{color} 

% Use doublespacing - comment out for single spacing
%\usepackage{setspace} 
%\doublespacing


% Text layout
\topmargin 0.0cm
\oddsidemargin 0.5cm
\evensidemargin 0.5cm
\textwidth 16cm 
\textheight 21cm

% Bold the 'Figure #' in the caption and separate it with a period
% Captions will be left justified
\usepackage[labelfont=bf,labelsep=period,justification=raggedright]{caption}

% Use the PLoS provided bibtex style
\bibliographystyle{plos2009}

% Remove brackets from numbering in List of References
\makeatletter
\renewcommand{\@biblabel}[1]{\quad#1.}
\makeatother

\usepackage{url}

% Leave date blank
\date{}

\pagestyle{myheadings}
%% ** EDIT HERE **


%% ** EDIT HERE **
%% PLEASE INCLUDE ALL MACROS BELOW

\newcommand{\pedro}[1]{\textcolor{black}{#1}}
\newcommand{\luciano}[1]{\textcolor{black}{#1}}

%% END MACROS SECTION

\begin{document}

% Title must be 150 characters or less
\begin{flushleft}
{\Large
\textbf{BraX-Ray: An X-Ray of the Brazilian Computer Science Graduate Programs}
}
% Insert Author names, affiliations and corresponding author email.
\\
Luciano A. Digiampietri$^{1,\ast}$, 
Jes\'us P. Mena-Chalco$^{2}$, 
Pedro O. S. Vaz de Melo$^{3}$,
Ana P. R. Malheiro$^{4}$,
D\^ania N. O. Meira$^{5}$,
Laryssa F. Franco$^{6}$,
Leonardo B. Oliveira$^{3}$
\\
\bf{1} School of Arts, Sciences and Humanities, University of S\~{a}o Paulo, Brazil
\\
\bf{2} Center for Mathematics, Computation and Cognition, Federal University of ABC, Brazil
\\
\bf{3} Computer Science Department, Federal University of Minas Gerais, Brazil
\\
\bf{4} Institute of Computing, University of Campinas, Brazil
\\
\bf{5} Institute of Computing, Fluminense Federal University, Brazil
\\
\bf{6} Faculty of Electrical Engineering and Computing, University of Campinas, Brazil
\\
$\ast$ E-mail: Corresponding digiampietri@usp.br
\end{flushleft}

% Please keep the abstract between 250 and 300 words
\section*{Abstract}
Research productivity assessment is increasingly relevant for allocation of research funds.
On one hand, this assessment is challenging because it involves both qualitative and quantitative analysis of several
characteristics, most of them subjective in nature. 
On the other hand, current tools and academic social networks make bibliometric data web-available to everyone for free.
Those tools, especially when combined with other data, are able to create a rich environment from which information on research productivity can be extracted. In this context, our work aims at characterizing the Brazilian Computer Science graduate programs and the relationship among themselves.  
We (i) present views of the programs from different perspectives, (ii) rank the programs according to each perspective and a
combination of them, (iii) show correlation between assessment metrics, (iv) discuss how programs relate to another, and (v)
infer aspects that boost programs' research productivity.
The results indicate that programs with a higher insertion in the coauthorship network topology also possess a
higher research productivity between 2004 and 2009.

% Please keep the Author Summary between 150 and 200 words
% Use first person. PLoS ONE authors please skip this step. 
% Author Summary not valid for PLoS ONE submissions.   
%\section*{Author Summary}

\section*{Introduction}\label{sec:intro}
Productivity assessments of research groups are increasingly relevant. There are limited funds to foment research
activities and a strong competitiveness to obtain part of it. An accurate research productivity assessment would
make possible 
to allocate funds meritocratically. The problem is that research productivity
assessment is indeed a daunting task. This is because it involves both
qualitative and quantitative analysis of several characteristics, most of them
subjective in nature. Besides, there is no consensus on the metrics to be used
in the analysis and then, depending on the chosen ones, the assessment may
produce quite different results. 

What sharply distinguishes Brazil from other countries is its natural and cultural diversity~\cite{enciclopedia-britanica}.
This diversity can also be found in science: there are a number of high standard graduate programs with unique
characteristics spread throughout the territory. This is particularly true in the field of Computer Science (CS) since there
are graduate programs of excellence in all of the five regions of the country.  On one hand, these peculiarities makes hardly
possible to characterize and assess programs. On the other hand, Brazil is one of few nations 
to have information on virtually all its publications openly available in the {\em World Wide Web} combined in a single web-based
system: The Lattes Platform.

Therefore, by getting together Lattes, other web-based
tools (e.g. Microsoft Academic Search and Google Scholar), and information accessible online about journal impact factors
(e.g.  Thompson's Journal Citation Reports and Scimago Journal Rank impact factors), it is possible to create a rich
environment of raw data from which information can be extracted to characterize graduate programs as well as the relationship
among themselves.

Although there is no consensual and precise measure for the complete analysis of graduate programs, some metrics are commonly
used~\cite{laender2008}, namely: (i) programs' goals; (ii) faculty members; (iii) students; (iv) intellectual production; and
(v) social insertion. These metrics are present in the graduate programs evaluation reports performed by the Brazilian
Coordination for the Improvement of Higher Education Personnel (CAPES) once every three years, when CAPES assigns a weight
(numerical value between 3 and 7) to each of them. 
According to CAPES, a program which weight is 7 excels in their respective fields worldwide. (CAPES is a public agency within the Brazilian Ministry of Higher
Education).

Table~\ref{tbl:top} presents the Brazilian CS 
graduate programs whose CAPES weight is either 6 or 7.






{\bf Contribution:} Our work aims at characterizing the Brazilian CS graduate
programs and the relationship among themselves. To be precise, we (i) present
views of the programs from different perspectives, (ii) rank the programs
according to both each perspective and to a combination of them, (iii) show the
correlation between assessment metrics, (iv) discuss how programs relate to
another, and (v) infer aspects that boost programs' research productivity.  To
do that, mainly two characteristics are explored: (i) the intellectual
productivity in terms of bibliographic production and (ii) the relationships
among programs in terms of academic social networks. Quantitative indices (e.g.
citation count as well as h- and g-index) that reflect the quality of
intellectual output were associated with those two characteristics.  Besides,
all information used in the characterization is
publicly available in the web.


The relevance of our work rests on the benefits of exploring different metrics and relationships to characterize the research
productivity of programs. To our knowledge, ours is the first work to look at the Brazilian graduate programs from so many
perspectives derived from either bibliographic productions or academic social networks. 



% You may title this section "Methods" or "Models". 
% "Models" is not a valid title for PLoS ONE authors. However, PLoS ONE
% authors may use "Analysis" 
\section*{Materials and Methods}
\label{sec:method}

In this work, we have assessed all the 37 existing CS academic graduate programs in Brazil presented in both 2004-2006 and
2007-2009 triennia. (Note we have not considered professional graduate programs in the study.)  The evaluation has been
carried out based on bibliographic production 
of professors of the programs during the period in question. Those professors have been identified via CAPES' reports and
the list has been manually validated through the use of the Brazilian National Form of Graduate
Programs\footnote{www.capes.gov.br/avaliacao/documentos-de-area-/3270}. Both the reports about the Brazilian graduate
programs and the curricula of the professors are openly-available in the Web as HTML files.
For each professor, we have acquired her/his Lattes curriculum and extracted her/his full bibliographic
productions published in academic journals and conferences. Based on this information, we have generated the academic
statistics.  The dataset contained 732 professors, 17,976 publications (13,926 conference papers and 4,050 journal papers), 7,583
co-authorship relationship pairs among professors, and 1,428 co-authorship relationship pairs among graduate programs.
Figure~\ref{fig:diagram} illustrates a schematic data flow diagram considered in our work. In what follows, we detail this
method, describe tools employed, and discuss statistics used.




% ---------------------------------------------------------------------------- %
\subsection*{Data Gathering}

Lattes is a web-based curriculum system that embraces the curricula (i.e.,
research productivity) of the major professionals and researchers working in Brazil. Lattes curricula have been designed to
show individual public information for every research registered on the system. In this context, performing a summarization
and evaluation of bibliographic production for a group of registered researchers requires a systematic effort.

In this work, we have used the bibliographic production registered on Lattes as our data source. To gather this information,
we first obtained programs' professor names from two CAPES's triennial reports (2004-2006 and 2007-2009). These reports, also
called in Portuguese {\em Cadernos de Indicadores}, and other CAPE's documents are available in the web.

With this process, 732 professors were identified and
associated with 37 CS academic graduate programs. These 37 programs are the ones presented in both triennia.

For each professor, we have gotten her/his corresponding Lattes ID (each ID is composed of 16 digits), which in turn make us
able to build the URL to access her/his curriculum online. Thus, a curriculum in HTML format was
obtained for each professor identified in the above process. This job has been carried out in a semi-automatic fashion through
the Lattes searching tool\footnote{\url{qualis.capes.gov.br/webqualis}}.

\subsection*{Data Parsing}

The HTML file from each curriculum is processed in two stages, namely (i) pre-processing and (ii) data extraction.  The
former removes the end-of-line characters, special characters and multiple blank spaces. This pre-processing is required to
make easier the identification of patterns -- which is performed in the next step. The latter is responsible for extracting
relevant information from each curriculum. 

To perform this extraction, a set of regular expressions ({\em regex}) is executed.  Initially, the expressions are used to
extract general and personal information, e.g., researcher's name and gender.  Later on, a regex is used to break the
curriculum into its main sections -- e.g., {\em Academic Degrees}; {\em Research Areas of Interest}; and {\em Bibliographic
Production}. 

Subsequently, for each section, we have employed tailor-made regex to extract information precisely, e.g., in the {\em Bibliographic Production Section}, there are regular expressions to identify each sort of production, such as journal or conference papers.
All items from each section is identified and organized according to their fields; for instance, journal
paper description comprises the following fields: authors, title, publication year, page numbers as well as journal name,
number, and volume. All information considered relevant is stored in an XML file, one from each curriculum. These files have
then been used to populate a local relational database.


\subsection*{Data Storage \& Enrichment}
\label{sec:enrich}

We have set up a relational database and automatically populated it with data from the XML files computed in the above
process. Like any other source of data manually populated, Lattes suffers from lack of standardization and typos. To deal
with these problems we have used dictionarization~\cite{okazaki2010} and approximate string matching strategies~\cite{cohen2003}.
Subsequently, we have enriched the database with third-party information on research productivity related to journal and
conference academic full papers. More precisely, we enriched our database with the following data:

\begin{enumerate}

        \item {\em Impact Factor:} We have used the well-known Thompson's Journal Citation Reports
		(JCR) 
		and Scopus's Scimago Journal Rank
            	(SJR) 
		impact factors. 

        \item {\em Citations:} We have considered citations of two different sources, namely Google
            Scholar\footnote{\url{scholar.google.com}} and Microsoft Academic Search\footnote{\url{academic.research.microsoft.com}}.

        \item {\em Indices:} We also have looked into how programs behavior face publication counts as well as  h- and g-index --
            the indices have been calculated based on Google Scholar and Microsoft Academic citation counts.
        
        \item{\em CAPES Qualis:} CAPES Qualis (or Qualis for short) is a set of criteria which CAPES uses to assess the
            Brazilian scientific production. From time to time, a new version of Qualis ranking is released assigning one of
            the following weights to each publication venues: A1, A2, B1, B2, B3, B4, B5, and C. In Qualis ranking system, A1
            is the highest weight while C is unvalued.  In order to compare only numerical measurements, numeric values have been assigned to each of the Qualis weight, \luciano{namely: A1=100, A2=85, B1=70, B2=50, B3=20, B4=10, B5=5, and C=0. Note this mapping from weights to numerical values are defined in the CAPES’ Computer Science Report.} Information about the current Qualis classification is accessible online\footnote{\url{qualis.capes.gov.br/webqualis}}. 


\end{enumerate}

\subsection*{Analysis}
\label{sec:analysis}

Once we had all the above information obtained/derived we started the analysis itself. The analysis have followed two
different lines: (i) productivity assessment and (ii) academic social network.  In the former, we presented views of
the programs from different perspectives and rank the programs accordingly. In the latter, we discussed the
evolution of the Brazil's research productivity as a whole.

For a given graduate program, we present its performance from, e.g., the JCR, SJR, and Qualis perspectives. We show its
citation count, h- as well as g-index for both Microsoft Academic Search and Google Scholar, and discuss how the results
of its assessment may vary when the perspective changes. Concerning social networks, we have considered programs as
nodes and drawn their collaboration also based on papers' coauthorship. 

The algorithm used to identify all co-authors of a publication is based on the comparison between publication titles obtained
from researchers/programs~\cite{mena-chalco2009}. Because of inconsistencies in filling in information in Lattes, the
comparison of any two publications is made through an approximate string matching between titles of papers.  In other words,
two papers are considered the same if their titles are at least 90\% similar. The similarity between them was measured using
the Levenshtein Distance~\cite{navarro2001string}. It is important to note that the co-authorship between the programs $i$
and $j$ is referred to coauthorship between professors associated to graduate programs $i$ and $j$, respectively. Professors
associated with two or more programs are not taken into account in our analysis. \luciano{A manually curated dataset was produced to evaluate the parser and the deduplication technique. This dataset contains information about 36 researchers and 620 publications. More than 99\% of the fields were correctly parsed and the accuracy of the coauthorship identification was above 99\% (with specificity above 99.9\%, sensibility above 88\% and F1 score about 94\%).}



\newcommand{\boy}{\textordmasculine}
% Results and Discussion can be combined.
\section*{Results}\label{sec:result}
In this section, we present results on program rankings and, subsequently, on
academic social network.





% ---------------------------------------------------------------------------- %
\subsection*{Program Ranking}
\label{sec:rp}

Table~\ref{tbl:ranking} shows the programs numbered according to CAPES' reports. (I.e., programs 1 to 7 in Table~\ref{tbl:ranking}, Column 1 correspond to the
programs listed in Table~\ref{tbl:top}.)
They are ranked by the various metrics described 
previously,
 namely: Microsoft Academic Search citation
count (MS CC) and Google Scholar citation count (Scholar CC) as well as h- and g-index based on them, impact factors (JCR and
SJR), and Qualis considering the bibliographic production from 2004 to 2009. Table's values present the ranking position of a
program for each of those metrics. 
It is worth noting that the positions of a
program may greatly vary depending on the ranking being used.

Figure~\ref{fig:position} and~\ref{fig:position2}  illustrate this dynamic as well, i.e., it summarizes how the different rakings affect the
programs' positions (Figure~\ref{fig:position}) and also shows how programs evolved over time
(Figure~\ref{fig:position2}).
In these figures, the programs in axis $x$ are sorted by the 
\textit{median rank} metric.

Observe, for instance, that
 program 7's position ranges from 6th to 37th (Figure~\ref{fig:position} and Table~\ref{tbl:ranking}) and the median difference in the program 7's positions between the triennia 2007-2009 and 2004-2006 is -2, i.e., program 7 is better ranked in the triennium 2007-2009~(Figure~\ref{fig:position2}). Besides, along the triennia, program 17's median ascended 11 positions and programs 2 and 18 have kept their median constant (Figure~\ref{fig:position2}). All in all, programs 2, 25, and, 5 presented the lowest median in both triennia, i.e., 2.5, 4, and 5.5, respectively. (Figure~\ref{fig:position}).
This concise representation allows us to observe the evolution (or not) of a given program in the context of Brazil.

We have also measured the correlation between the various rankings
(Figure~3) and their correlation face the median ranking value
(Figure~4). We highlight  the correlation among three groups:
(i) MS g-index, Scholar g-index, MS h-index, Scholar h-index; (ii) JCR, SJR, Qualis; and
(iii) Pubs. count, MS CC, and Scholar CC.

According to Figure~4, the most representative rankings
are the ones based on citations, namely Scholar CC and MS CC; both exhibiting a
correlation greater than 90\%. 
In fact, Scholar's correlation coefficient was slightly
higher than MS's, reason why we have employed the Scholar CC ranking later 
 in the next section
as base for the comparison with network metrics.


\subsection*{Network Analysis}
\label{sec:network}

So far in this paper 
we have shown how traditional productivity metrics affect program ranking. In
this section we move away from these traditional metrics and start to look at the academic social network formed by the programs
and the way they collaborate among themselves. 

\subsubsection*{Network Formation}

More specifically, we carried out a Social Network Analysis (SNA) over the academic social network made up by the programs.
This is a particular type of social network in which the {\em nodes} represent the programs and the {\em edges} indicate that
the programs (i.e., one or more of their professors) collaborated and published at least one paper together. Collaboration
networks have been widely analyzed~\cite{newman2004cna}, as these studies disclose several interesting features about
academic communities that comprise them.

Here we have built two sorts of collaboration networks, namely {\em undirected network}
($G_U(V,E_U)$) and {\em directed network}  ($G_D(V,E_D)$); both of them having programs as the
set of vertices (or nodes) $V$ of a graph.
In the former, an edge exists between two nodes if the programs they represent have
published at least one paper together. 
In the latter, a directed edge $(i\rightarrow j)$ exists if programs $i$ and $j$ have
coauthored a paper and the paper's first author is affiliated to program $i$. Here we
assume the first author is the paper's main author and the other authors are
researchers who supported her/him in the work. 
I.e., the set of edges $E_D$ from $G_D$
potentially maps the ``needs help from'' relationship between researchers. \pedro{Regarding this particular network, it is worth pointing out that we are only considering papers in which the first author is a professor affiliated to a Brazilian CS program.}

\subsubsection*{Network Metrics}

Now we describe the network metrics we use to infer the productivity of the programs. As
we will see, through these metrics we were able to find out clusters of programs and, more
importantly, to highlight the program's roles in the CS production in Brazil.

\par{\textbf{Centrality in the network $G_U$.}} 
Node centrality is performed by using three metrics: degree (Cnt.Deg), betweenness (Cnt.Bet), and closeness (Cnt.Clos)
 centralities~\cite{Bonacich:1987:AFM}. These metrics aim at identifying nodes that are strategically
situated within the network's topology. A strategic location in a network may indicate that a node
has a higher influence or even hold the attention of nodes that occupy positions that are not as
socially relevant as its. 

\par{\textbf{Clustering Coefficient of the network $G_U$.}}
The clustering coefficient (Cl.Coef) of a vertex $i$ is regularly used to measure how clustered a group of vertices
is~\cite{vazdemelo:2008a}. In the case of $G_U$, a low clustering coefficient likely means that the program has a wide
network of collaborations, not being limited by geographical or any other constraints.

\par{\textbf{PageRank of the network $G_D$.}} 
We have used the Google's PageRank algorithm to identify important nodes in directed networks by
recursively transferring a node's importance to other nodes that the former ``considers'' important~\cite{langville2009google}. In the case of $G_D$, we apply the PageRank algorithm to identify those programs which are more requested for help in papers.

In Figures~5 and~6, we show the two coauthorship networks among graduate programs we construct in this paper.
Node colors represent the geographic region which the program is located, namely north (blue), northeast (red), central-west
(purple), southeast (green), and south (gray). 
Node sizes are proportional to their degrees for $G_U$ and to their PageRank for $G_D$.  Again, we label the top seven
programs according to CAPES (see Table~\ref{tbl:top} and Column 1 of Table~\ref{tbl:ranking}). These graphs are drawn using a
force-direct layout algorithm (FDLA)~\cite{fruchterman:1991}, which tries to minimize the number of crossing edges. In other
words, this layout highlights communities of nodes, i.e., nodes with a high number of common neighbors are placed closer.

First, it is worth noting that the top programs are clearly visible in both networks,
i.e., they have high values of degree and PageRank, what indicates that these two network
metrics are able to highlight them. Moreover, observe that there is no clear community or
cluster among the programs, although there is a slight tendency of geographically closer
programs to collaborate more. The geographic assortativity~\cite{Newman2003a} considering
the region of each program is $0.19$. On the other hand, the degree assortativity is
negative for both networks, indicating that there is also a small tendency of programs
with different degree magnitudes to connect more.  This is expected, since it is a common
practice of smaller programs to collaborate with bigger ones. This is because a significant part of
the professors of the former have graduated in the latter. In Table~\ref{tbl:netcharac} we
describe these and other characteristics of the networks $G_U$ and $G_D$.




We believe that the aforementioned node's features are able to capture a great part of the
programs collaboration dynamics. While the centrality metrics indicate the ``importance''
of the program in the network, the clustering coefficient shows how broad are the node's
connections. Additionally, the PageRank metric is able to capture a certain degree of
hierarchy among the programs, pointing which program receives more ``help'' requests. 

In order to further investigate these features' potential to discover knowledge, we rank the programs according to each of the aforementioned  network metrics, and three other metrics, i.e., the program's age, the program's size ({\em Profs. \#}) and the program's Scholar CC productivity index. We show in Figure~\ref{fig:correlation-networks} the Spearman's Rank Correlation $\rho$ among these ranks. If, for instance, the correlation $\rho$ between the rank generated by the degree centrality and the Scholar CC is $1$, then the degree centrality generates the same rank of programs the Scholar CC generates. If $\rho$ is $-1$, then it generates the complete opposite rank.

First, note how the network metrics have positive rank correlation with the Scholar CC.
Additionally, PageRank is the metric most correlated with the Scholar CC, what
corroborates to our assumption that in papers involving different programs, the first author
usually seeks for aid in other programs, creating the so called ``needs help
from''
hierarchy in the network.



\subsubsection*{Network-based Classification}
\label{sec:class}

As we observed in Figures~7 and~8, simple metrics are able to highlight the most important nodes in the network. However, observe how these metrics produce very different results and fail to separate the top 7 programs according to CAPES from several other nodes, which have apparently similar importance in the network. 

Thus, in order to verify if network metrics are able to clearly separate these top programs from the rest, we analyze the principal components of the feature matrix formed from
these metrics. Principal Component Analysis (PCA)~\cite{pca} is a widely used statistical
technique for unsupervised dimension reduction. It transforms the data into a new
coordinate system such that the greatest variance is achieved by projecting the data into
the first coordinate, namely principal component, the second greatest variance achieved
into the second coordinate, the second component, and so on. 

In Figure~\ref{fig:pca}, we show the first two principal components of the matrix formed
from the network features. These two components account for approximately $95\%$ of the
variation. It is fascinating that these new dimensions are able to clearly cluster the Brazil's
top programs according to CAPES (labeled in the figure).  Note that the first dimension,
which accounts for approximately $77\%$ of the variation, is more related to the node
importance in the network, since the component coefficients of the centrality metrics and
the PageRank are significantly positive. On the other hand, the second component, which
accounts for approximately $18\%$ of the variation, is more related to the clustering and
collaboration dynamics of the programs. \pedro{Note that it is able to discriminate well programs located in the left far side of the figure, indicating that a program should also avoid a collaboration strategy that leads to either very high or very low values for the clustering coefficient.} To illustrate that, consider the two most far
points in this dimension, both marked with a square symbol. While the one with the most
positive value has a degree of 6 and a clustering coefficient of $\approx 0.94$, the most
negative one has a degree of 1 and a clustering coefficient of $0$ (by definition).

\subsubsection*{Deconstructing the Collaborations}

We have shown so far that network metrics have significant correlation to 
research productivity in our context. In this section, in turn, we go
further in this analysis by looking into the reasons why an edge remains persistent (kept alive) over the years.
Again, we have considered the period between 2004 and 2009, divided into two triennia (i.e.
2004-2006 and 2007-2009). We consider that an edge $(i,j)$ is persistent if program $i$ collaborated
with program $j$ in both triennia. Otherwise, we call this edge non-persistent.

In Figure~10, we show the total number of edges that are persistent and
non-persistent grouped by six different metrics, namely distance, max(Age), min(Age), max(Scholar
CC), min(Scholar CC), and PageRank.
More precisely, in Figure~10A, we group the edges $(i,j)$ by the geographic distance
between nodes $i$ and $j$. Note that the fraction of non-persistent edges grows significantly as the
distance grows, what indicates that distance is a determinant factor for an edge to persist or not. 

Figure~10B and Figure~10C, in turn, shows that the program's age also influences
the edges persistence. They show, respectively, 
the age of the older -- max(Age) --  and the younger -- min(Age) --  node of the edges.
Note that the proportion of persistent edges is significant only when max(Age) is higher than 30.
Also, note that as min(Age) grows, the proportion of persistent edges grows as well, indicating that
whenever the edge is between two old (well-established) nodes, the edge is more likely to be
persistent.

In addition to nodes distance and age, we also studied how edge persistence varies as a function of
nodes productivity. 
In Figure~10D and in Figure~10E,  we show the number of persistent and non-persistent edges 
grouped by the Scholar CC value of the most -- max(Scholar CC) --  and the least -- min(Scholar CC)
-- productive node that comprises edges.
First, note that while it is very unlikely to have a persistent edge when max(Scholar CC) is very
low, it is very unlikely to have a non-persistent edge when max(Scholar CC) is high
(Figure~10D). Besides, from Figure~10E, we can note that the proportion of
non-persistent edges drops significantly as min(Scholar CC) grows. All these observations suggest
that the node productivity is a key factor for the persistence of its edges. 

Now, in Figure~10F, we consider a network metric, i.e.,  we show the number of
edges grouped by the minimum PageRank value (min(PageRank)) of their nodes. Observe that the
proportion of non-persistent edges drops significantly as the min(PageRank) of the node grows. This
suggests that edges formed from nodes with high PageRank values are likely to be persistent.  In other words, when programs
which are frequently providing ``help'' to other programs (i.e. high PageRank programs) collaborate
among themselves, it is very likely that the ``help'' will come in a bidirectional way, reinforcing
the collaboration and, as a consequence, the edge persistence. 

Finally, observe how the histograms of Figures~10B,~10C
and~10D are bimodal, showing two well defined masses of data. This fact corroborates
with the cluster analysis we showed in Figure~\ref{fig:pca}. 

All in all, it looks like that the edge formation process is governed by at least three processes
that splits the edges (and the programs, as we see in Figure~\ref{fig:pca}) into groups with
different characteristics. We conjecture that these edge creation processes are the following:

\begin{itemize}
  \item $(small\leftrightarrow small)$. Occurs when new or low-productive programs collaborate among
      themselves. The collaboration may have started, for instance, when two ex-colleagues graduated
      together and kept collaborating after they started to work for different new programs. These
      edges are more likely to not persist.
  \item $(small\rightarrow BIG)$. Occurs when a new or low-productive program seeks a collaboration
      with well-established ones. These edges are created when, for instance, a professor of a new
      program graduated in an established one and continued to collaborate with her/his former
      advisor. These edges are either like to persist or not.
   \item $(BIG \leftrightarrow BIG)$. Occurs when well-established programs collaborate among
       themselves. It is common that experts from well-established programs are well known by the
       academic community and seek each other's ``help'' in a bidirectional collaboration. These
       edges are more likely to persist.
\end{itemize}



% ---------------------------------------------------------------------------- %

\section*{Related Work}
\label{sec:rw}


Recently, the number of works focused on research productivity assessment has grown considerably (e.g.
\cite{hirsch2005,bollen2006,garfield2006,mena-chalco2009,duffy2011,martins2010}
). Most of them, relies on metrics such as
 Impact Factor~\cite{garfield1955,garfield2006}, h-index~\cite{hirsch2005}, and
citation count to assess productivity in a certain area of research. Duffy, Jadidian, and Webster~\cite{duffy2011}, for
instance, carried out an evaluation of academic productivity within psychology. Their work has been based on h-index,
citation counts, and author-weighted publication counts. Further, they have analyzed the impact of the gender and tenure
on researchers' productivity. 

Martins \emph{et al.}\cite{martins2010}, in turn,  have assessed the quality of conferences based on citation count.
They have also pointed out the need for new metrics and come up with some of them exclusively tailored to assess
conferences.

There are also works that combine SNA and the CS field (e.g.
\cite{Menezes09ageographical,franceschet2010comparison,franceschet2011collaboration}). Among them, Menezes \emph{et
al.}~\cite{Menezes09ageographical} analyzed the CS research productivity in different regions of the world (Brazil,
North America, and Europe) using collaboration networks. They presented the evolution of CS subfields for the period
1994-2006 and the inter-relationship between the CS subfields. They have also discussed the research productivity in
each world region; and contrasted the regional networks' idiosyncrasies.

Franceschet~\cite{franceschet2011collaboration}, in turn, collected data from the DBLP CS Bibliography~\cite{ley2002} to
study coauthorship networks and analyzed academic collaboration in CS. Franceschet have shown that the collaboration
level in CS papers is rather moderate compared to other fields. Their results also indicates that conferences  can
communicate results quicker, while journals can make relationships stronger.

Like ours, some works have also narrowed even further their object of study and focused on the Brazil's CS research
productivity. 
For instance, Laender \emph{et al.}~\cite{laender2008} have assessed the excellence of the top Brazilian CS graduate
programs. They have contrasted Brazilian programs against reputable programs in both North America and Europe and
conclude that the CS field in Brazil has reached the maturity. This study has been based on recent data from
DBLP. 

Digiampietri and Silva~\cite{digiampietri2011}, in turn, introduced a framework for social network analysis and visualization that allows
users to access relevant information about research groups using web-available data. The framework searches curricula in
Lattes, extract relevant information, identify the relationships among authors, and then builds a social network.  
In the work, they have used the field of CS in Brazil as a case of study.

Finally, Figueiredo and Freire~\cite{freire2011} have presented a study of the Brazil's CS academic social network \pedro{based on the DBLP}.  In
this network, they have noticed the existence of super peers, i.e., that a small number of nodes presented a very high
degree (some researchers collaborated with many other researchers), while the great majority of nodes had a lower
degree. They have also come up with a metric\pedro{, namely \textit{degree-cut-weight},} to classify individuals in collaborative networks, where the
importance of a node in a group is proportional to the intensity of their relationships with nodes from
another group. They then applied this and other metrics to rank graduate programs and compared the
results with CAPES ranking.

\pedro{Likewise the aforementioned work~\cite{freire2011}, ours studies the Brazil's CS academic social network, as well. However, our approach is different. See, first we collect our dataset from Lattes, to our knowledge, the most reliable base of the Brazilian scientific production. Second, while Figueiredo and Freire aim at evaluating CS programs using three metrics (namely degree-cut-weight as well as number of publications and collaborators), our goal is to assess them from ten different perspectives based on quantitative ranking systems. Finally, different from~\cite{freire2011}, we isolate network metrics from quantitative metrics to show that network metrics alone can spot the most productive CS programs in Brazil.} 


% ---------------------------------------------------------------------------- %

\section*{Conclusion}\label{sec:conclusion}

Research productivity assessment is important because it allows allocating the limit funds to foment research activities in a
meritocratic way. However, the fact that the assessment involves both qualitative and quantitative analyses of several
characteristics -- most of them subjective in nature -- turns this assessment challenging.  Besides, depending on the metrics
used to carry out the assessment, results change. 

In this work, we aim at presenting an X-ray of Brazilian CS graduate program's research productivity. To be precise,  we have
shown views of the programs from different perspectives such as h- and g-index as well as citation counts. To do that, we have
explored mainly two characteristics: (i) the intellectual productivity and (ii) the academic social network for the period
between 2004 and 2009. The results indicate that programs better located in the network topology are more productive.  We
believe that the obtained results are paramount for assessing academic performance, getting current collaborations stronger, and
pointing out new partnerships among programs. 

\luciano{Despite our study were able to draw precise conclusions, future directions are still possible. For instance, one could adapt our methodology and thus apply it to assess other Graduate Progress, whether in Brazil or not. Besides, our methodology could be improved by also taking into consideration the roles that funds and governmental incentives play in the scientific production.}


% ---------------------------------------------------------------------------- %

\section*{Acknowledgments}
Authors would like to thank Mauro Zackiewicz for his support in the conception of this work. We would like to also thank Andr\'{e} Drummond, Daniel Lapolla, Jo\~{a}o Furtado, Luiz Celso, and Thiago Godoi for their valuable support during this work.



%\section*{References}
% The bibtex filename
\bibliography{refs}

\section*{Figure Legends}


\begin{figure*}[!ht]
    %\includegraphics[width=0.79\textwidth]{figure1}
    \caption{{\bf Schematic data flow diagram of the proposed method.}
	Web-available data sources are represented by clouds.
	Processes are represented by blocks in gray color.
    Each arrow represents the information flow between processes/data sources.} 
    \label{fig:diagram}
\end{figure*}


\begin{figure}[!ht]
%\includegraphics[width=0.498\textwidth]{figure02.tiff}
\caption{
{\bf Boxplots of the programs' positions for the various rankings and evolution of programs in time.}  
}
\label{fig:position}
\end{figure}

\begin{figure}[!ht]
%\includegraphics[width=0.498\textwidth]{figure03.tiff}
\caption{
{\bf Difference between the rankings in the triennia 2007-2009 and 2004-2006.}
}
\label{fig:position2}
\end{figure}


\begin{figure}[!ht]
\label{fig:correlation-among}
%\includegraphics[width=0.4\textwidth]{figure04.tiff}
\caption{{\bf Correlation among the rankings.}}
\end{figure}

\begin{figure}[!ht]
\label{fig:correlation-median}
%\includegraphics[width=0.3\textwidth]{figure05.tiff}
\caption{ {\bf Spearman's correlation between each ranking and its median value.}}
\end{figure}


\begin{figure}[!ht]
    %\includegraphics[width=0.4\textwidth]{figure06.tiff}    
    \caption{ {\bf Spearman's Rank Correlation for different metrics.} }
    \label{fig:correlation-networks}
\end{figure}

\begin{figure}[!ht]
  \label{fig:undirected}
  %\includegraphics[width=0.45\textwidth]{figure07.tiff}
	\caption{ {\bf Undirected co-authorship network belonging to 37 Brazilian CS graduate programs.} The
  programs are represented by nodes, the co-authorships by edges, and node
  sizes are proportional to their degree}  
\end{figure}	

	\begin{figure}[!ht]
  \label{fig:directed}
  %\includegraphics[width=.45\textwidth]{figure08.tiff}
	\caption{ {\bf Directed co-authorship network belonging to 37 Brazilian CS graduate programs.} The
  programs are represented by nodes, the co-authorships by edges, and node
  sizes are proportional to their PageRank}  
  \end{figure}


\begin{figure}[!ht]
    %\includegraphics[width=0.51\textwidth]{figure09.tiff}    
    \caption{{\bf The first two principal components of the matrix formed from the network features.}}
    \label{fig:pca}
\end{figure}


\begin{figure}[!ht]
    %\includegraphics[width=0.51\textwidth]{figure10.tiff}    
    \caption{{\bf The number of edges which are persistent and non-persistent according to various metrics.} }
    \label{fig:linkanalysis}
\end{figure}


\section*{Tables}

\begin{table}[!ht]
\caption{
\bf{Top Brazilian CS Graduate Programs according to CAPES.}}
    \begin{tabular}{ c ll }
    \hline
    {\bf Prog. \#}  & {\textbf University } & {\textbf Institution/Department} \\ \hline
	1                  & PUC-RIO              & Department of Informatics \\
	2                  & UFMG                 & Computer Science Department\\
	3                  & UFRJ                 & ALC Inst. and G. Sch. of Res. and Eng.  \\ 
	4                  & UFPE                 & Center of Informatics \\
	5                  & UFRGS                & Institute of Informatics\\
	6                  & UNICAMP              & Institute of Computing\\
	7                  & USP/SC               & Institute of Math. Science and Comp. \\  \hline
\end{tabular}
\label{tbl:top}
\end{table}

\begin{table}[ht!]
\scriptsize
\caption{\bf{Brazilian CS graduate programs ranked using different metrics.}}
	\renewcommand{\arraystretch}{0.95}
	\setlength{\tabcolsep}{0.5em}
\label{tbl:ranking}
\begin{tabular}{ c  r  r  r  r  r  r  r  r  r  r  r  r  r }
\hline
{\bf CAPES} & {\bf MS} & {\bf Scholar} & {\bf JCR} & {\bf SJR} & {\bf Qualis} & \multicolumn{1}{c}{{\bf MS}}     &\multicolumn{1}{c}{{\bf MS}}      & {\bf Scholar} & {\bf Scholar} & {\bf Pubs.} & {\bf Best} & {\bf Worst} & {\bf Median} \\
{\bf Prog. \#}    & {\bf CC} & \multicolumn{1}{c}{{\bf CC}}      &           & &              & {\bf h-index} & {\bf g-index} & {\bf h-index} & {\bf g-index} & {\bf count} & \multicolumn{1}{c}{{\bf rank}} &\multicolumn{1}{c}{{\bf rank}}
&\multicolumn{1}{c}{{\bf rank}}\\ \hline
1           & 2\boy    & 2\boy         & 13\boy    & 11\boy    & 10\boy       & 5\boy         & 6\boy         & 6\boy         & 5\boy         & 6\boy       & 2\boy      & 13\boy      & 6\boy \\ 
2           & 1\boy    & 1\boy         & 2\boy     & 1\boy     & 4\boy        & 3\boy         & 2\boy         & 3\boy         & 3\boy         & 7\boy       & 1\boy      & 7\boy       & 2.5\boy \\ 
3           & 5\boy    & 6\boy         & 4\boy     & 2\boy     & 5\boy        & 22\boy        & 26\boy        & 20\boy        & 26\boy        & 4\boy       & 2\boy      & 26\boy      & 5.5\boy \\
4           & 6\boy    & 7\boy         & 10\boy    & 5\boy     & 2\boy        & 27\boy        & 28\boy        & 31\boy        & 32\boy        & 5\boy       & 2\boy      & 32\boy      & 8.5\boy \\
5           & 3\boy    & 3\boy         & 5\boy     & 3\boy     & 6\boy        & 23\boy        & 24\boy        & 26\boy        & 27\boy        & 2\boy       & 2\boy      & 27\boy      & 5.5\boy \\ 
6           & 4\boy    & 4\boy         & 3\boy     & 4\boy     & 8\boy        & 17\boy        & 15\boy        & 18\boy        & 19\boy        & 12\boy      & 3\boy      & 19\boy      & 10\boy \\ 
7           & 18\boy   & 11\boy        & 7\boy     & 6\boy     & 7\boy        & 37\boy        & 36\boy        & 35\boy        & 36\boy        & 9\boy       & 6\boy      & 37\boy      & 14.5\boy \\
8           & 11\boy   & 8\boy         & 8\boy     & 15\boy    & 12\boy       & 14\boy        & 17\boy        & 13\boy        & 15\boy        & 13\boy      & 8\boy      & 17\boy      & 13\boy \\ 
9           & 14\boy   & 15\boy        & 1\boy     & 13\boy    & 3\boy        & 24\boy        & 22\boy        & 27\boy        & 23\boy        & 34\boy      & 1\boy      & 34\boy      & 18.5\boy \\ 
10          & 13\boy   & 17\boy        & 12\boy    & 7\boy     & 9\boy        & 9\boy         & 12\boy        & 11\boy        & 13\boy        & 14\boy      & 7\boy      & 17\boy      & 12\boy \\ 
\hline
11          & 12\boy   & 12\boy        & 14\boy    & 9\boy     & 14\boy       & 8\boy         & 14\boy        & 9\boy         & 14\boy        & 17\boy      & 8\boy      & 17\boy      & 13\boy \\
12          & 10\boy   & 9\boy         & 9\boy     & 16\boy    & 13\boy       & 4\boy         & 4\boy         & 8\boy         & 4\boy         & 31\boy      & 4\boy      & 31\boy      & 9\boy \\ 
13          & 23\boy   & 24\boy        & 15\boy    & 17\boy    & 21\boy       & 25\boy        & 19\boy        & 22\boy        & 20\boy        & 32\boy      & 15\boy     & 32\boy      & 21.5\boy \\ 
14          & 9\boy    & 10\boy        & 25\boy    & 18\boy    & 17\boy       & 7\boy         & 7\boy         & 4\boy         & 7\boy         & 11\boy      & 4\boy      & 25\boy      & 9.5\boy \\ 
15          & 25\boy   & 25\boy        & 31\boy    & 33\boy    & 29\boy       & 28\boy        & 31\boy        & 24\boy        & 28\boy        & 28\boy      & 24\boy     & 33\boy      & 28\boy \\ 
16          & 27\boy   & 33\boy        & 17\boy    & 29\boy    & 30\boy       & 19\boy        & 16\boy        & 29\boy        & 25\boy        & 36\boy      & 16\boy     & 36\boy      & 28\boy \\ 
17          & 26\boy   & 26\boy        & 21\boy    & 23\boy    & 22\boy       & 32\boy        & 27\boy        & 34\boy        & 31\boy        & 26\boy      & 21\boy     & 34\boy      & 26\boy \\ 
18          & 19\boy   & 21\boy        & 11\boy    & 10\boy    & 18\boy       & 31\boy        & 32\boy        & 32\boy        & 35\boy        & 21\boy      & 10\boy     & 35\boy      & 21\boy \\ 
19          & 17\boy   & 16\boy        & 24\boy    & 8\boy     & 11\boy       & 15\boy        & 18\boy        & 15\boy        & 16\boy        & 19\boy      & 8\boy      & 24\boy      & 16\boy \\
20          & 28\boy   & 27\boy        & 16\boy    & 22\boy    & 25\boy       & 35\boy        & 37\boy        & 37\boy        & 37\boy        & 18\boy      & 16\boy     & 37\boy      & 27.5\boy \\ 
\hline
21          & 21\boy   & 19\boy        & 27\boy    & 24\boy    & 23\boy       & 21\boy        & 29\boy        & 21\boy        & 24\boy        & 16\boy      & 16\boy     & 29\boy      & 22\boy \\ 
22          & 34\boy   & 34\boy        & 26\boy    & 28\boy    & 34\boy       & 30\boy        & 34\boy        & 28\boy        & 33\boy        & 33\boy      & 26\boy     & 34\boy      & 33\boy \\ 
23          & 24\boy   & 28\boy        & 22\boy    & 26\boy    & 26\boy       & 13\boy        & 11\boy        & 19\boy        & 18\boy        & 25\boy      & 11\boy     & 28\boy      & 23\boy \\
24          & 20\boy   & 20\boy        & 20\boy    & 20\boy    & 19\boy       & 12\boy        & 21\boy        & 10\boy        & 21\boy        & 10\boy      & 10\boy     & 21\boy      & 20\boy \\
25          & 7\boy    & 5\boy         & 6\boy     & 12\boy    & 1\boy        & 2\boy         & 5\boy         & 2\boy         & 2\boy         & 3\boy       & 1\boy      & 12\boy      & 4\boy \\ 
26          & 16\boy   & 18\boy        & 23\boy    & 21\boy    & 20\boy       & 10\boy        & 8\boy         & 14\boy        & 10\boy        & 20\boy      & 8\boy      & 23\boy      & 17\boy \\ 
27          & 29\boy   & 31\boy        & 18\boy    & 25\boy    & 15\boy       & 6\boy         & 9\boy         & 7\boy         & 9\boy         & 22\boy      & 6\boy      & 31\boy      & 16.5\boy \\ 
28          & 22\boy   & 22\boy        & 19\boy    & 19\boy    & 16\boy       & 11\boy        & 10\boy        & 5\boy         & 8\boy         & 29\boy      & 5\boy      & 29\boy      & 17.5\boy \\ 
29          & 15\boy   & 14\boy        & 30\boy    & 14\boy    & 24\boy       & 1\boy         & 1\boy         & 1\boy         & 1\boy         & 1\boy       & 1\boy      & 30\boy      & 7.5\boy \\ 
30          & 35\boy   & 35\boy        & 35\boy    & 31\boy    & 28\boy       & 29\boy        & 30\boy        & 30\boy        & 29\boy        & 23\boy      & 23\boy     & 35\boy      & 30\boy \\ 
\hline
31          & 32\boy   & 30\boy        & 28\boy    & 32\boy    & 36\boy       & 33\boy        & 23\boy        & 33\boy        & 22\boy        & 37\boy      & 22\boy     & 37\boy      & 32\boy \\ 
32          & 8\boy    & 13\boy        & 34\boy    & 30\boy    & 27\boy       & 20\boy        & 3\boy         & 25\boy        & 6\boy         & 8\boy       & 3\boy      & 34\boy      & 16.5\boy \\ 
33          & 30\boy   & 23\boy        & 37\boy    & 36\boy    & 31\boy       & 18\boy        & 25\boy        & 12\boy        & 17\boy        & 15\boy      & 12\boy     & 37\boy      & 24\boy \\
34          & 31\boy   & 32\boy        & 33\boy    & 27\boy    & 35\boy       & 16\boy        & 13\boy        & 17\boy        & 11\boy        & 24\boy      & 11\boy     & 35\boy      & 25.5\boy \\ 
35          & 36\boy   & 36\boy        & 32\boy    & 35\boy    & 33\boy       & 34\boy        & 33\boy        & 36\boy        & 34\boy        & 30\boy      & 30\boy     & 36\boy      & 34\boy \\ 
36          & 33\boy   & 29\boy        & 29\boy    & 34\boy    & 32\boy       & 26\boy        & 20\boy        & 16\boy        & 12\boy        & 27\boy      & 12\boy     & 34\boy      & 28\boy \\ 
37          & 37\boy   & 37\boy        & 36\boy    & 37\boy    & 37\boy       & 36\boy        & 35\boy        & 23\boy        & 30\boy        & 35\boy      & 23\boy     & 37\boy      & 36\boy \\ \hline
\end{tabular}
\end{table}


\begin{table}[!ht]
\caption{\bf{Characteristics of the networks.}}
\label{tbl:netcharac}
    \setlength{\tabcolsep}{0.50em}
    \begin{tabular}{ lll }
    \hline
    \multicolumn{1}{c}{\textbf{Metric}}        & $G_U$      & $G_D$                         \\
    \hline
    Density                & 0.25       & 0.18                          \\
    Average Degree         & 11.4 (7.3) & in: 8.2 (5.9), out: 8.2 (5.4) \\
    Clustering Coefficient & 0.56       & -                             \\
    Diameter               & 3          & $\infty$                      \\
    Average Distance       & 1.9        & 1.9                           \\
    Degree Assortativity   & -0.23      & -0.27                         \\
    Region Assortativity   & 0.18       & 0.18                          \\
    \hline
    \end{tabular}
\end{table}

\end{document}
