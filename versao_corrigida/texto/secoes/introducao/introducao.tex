\chapter{Introdu\c{c}{\~a}o, justificativa e objetivos}\label{INTRODUCAO}
\begin{flushright}
	\textit{``Perdoe-me, meu amigo, não pelo o que eu fiz. Mas pelo que estou prestes a fazer.''\\
	(Ragnar Lothbrok)}
\end{flushright}

O termo \emph{e-Science} se refere à ciência que é realizada com o uso intensivo de computadores. Em projetos da \emph{e-Science} existe uma forte relação entre computação e outras áreas do conhecimento (biologia, astronomia, física, entre outras), tal que a primeira fornece ferramentas fundamentais para o sucesso de experimentos científicos computacionais da segunda. Um dos objetivos dessas ferramentas é ocultar detalhes técnicos computacionais, permitindo aos cientistas gerenciarem experimentos com maior facilidade \cite{Deelman2009}.

Uma das ferramentas para auxiliar na criação/manutenção de experimentos científicos computacionais são os sistemas gerenciadores de \emph{workflows} científicos (SGWCs). Segundo \citeonline{DigiampietriTese2007} e \citeonline{Wang2010}, \emph{workflows} científicos são processos estruturados e ordenados, construídos de forma manual, semiautomática ou automática e que permitem solucionar problemas científicos por meio de sua execução.

Um \emph{workflow} denota a execução controlada de diversas atividades em um ambiente potencialmente distribuído. \emph{Workflows} representam um conjunto de atividades a serem executadas, suas relações de interdependência, entradas e saídas~\cite{Medeiros2005}. Atividades são as unidades básicas de um workflow e podem ser serviços Web, métodos em alguma linguagem de programação, etc.

Três grandes projetos de \emph{e-Science} e seus desafios computacionais, solucionados com o auxílio de SGWCs, são enumerados por \citeonline{Olabarriaga2014}. O primeiro é a visualização de grandes quantidades de dados Astrofísicos presentes no projeto internacional IVOA\footnote{http://www.ivoa.net}, o qual tem experimentos com petabytes de dados a serem visualizados. O segundo são cálculos matemáticos em ambientes distribuídos como no projeto internacional HELIO\footnote{http://helio-vo.eu} criado por heliofísicos. O último são simulações genéticas em centros médicos de pesquisa como na universidade de Amsterdam \cite{Olabarriaga2014}.

Um \emph{workflow científico} modela um experimento científico construído por meio de diversas atividades conectadas que realizam uma tarefa computacional. Alguns SGWCs permitem que seja armazenado o \emph{log} de modelagem e execução do \emph{workflow} junto com seus parâmetros de execução, o que facilita sua execução por outros cientistas. As atividades usadas podem ser trechos de código fonte em Java (ou outra linguagem), aplicativos locais, serviços \emph{web} ou outros \emph{workflows} que foram encapsulados para esconder os detalhes computacionais, como a lógica de programação. Isso permite aos cientistas, sem grandes conhecimentos computacionais, ``programarem'' computadores para realizar seus experimentos sem se preocupar com detalhes de computação.

Atualmente, há um grande número de atividades disponíveis em repositórios como \emph{myExperiment} \cite{ROURE2015}, que armazena mais de \(2.500\) \emph{workflows} e \emph{BioCatalogue} \cite{Biocatalogue} que disponibiliza \(2.464\) serviços de bioinformática. A existência desse grande número de atividades acarreta em um grande potencial de reuso~\cite{Wang2010}, motivando a construção de sistemas para recomendar atividades aos cientistas durante a composição dos \emph{workflows}. Estes sistemas, possibilitam reduzir a criação de novas atividades redundantes e o tempo total para a construção de \emph{workflows}.

A recomendação de atividades em \emph{workflows} possui algumas peculiaridades que justificam o desenvolvimento de novas técnicas (para recomendar atividades) de recomendação ou a extensão das existentes. Essas peculiaridades são: i) a esparsidade dos dados, isto é, cada  \emph{workflow} é tipicamente composto por poucas atividades e as bases de  \emph{workflows}, muitas vezes, contêm mais atividades do que  \emph{workflows}; ii) a dependência entre as atividades dos  \emph{workflows}: ao contrário da recomendação de itens como músicas, a ordem em que as atividades serão executadas é extremamente importante para a correta criação de um  \emph{workflow}; e iii) a diversidade da representação, documentação ou anotação das atividades e  \emph{workflows}: existem poucas atividades com descrições detalhadas (incluindo algumas descrições formais e anotações ontológicas), enquanto outras possuem apenas a definição dos tipos de dados usados na entrada e o tipo de dado produzido como saída da atividade.

Este mestrado apresenta três contribuições principais. A primeira é uma técnica mista para recomendar atividades em  \emph{workflows} baseada em frequência de atividades e ontologia de domínio. A segunda é a modelagem da recomendação de atividades como um problema de classificação, regressão, classificação composta (que utiliza dos resultados de outros experimentos para classificar) e um \emph{ensemble} de classificadores. A terceira é a comparação do desempenho de diferentes técnicas de recomendação de atividades em  \emph{workflows} científicos utilizando-se uma mesma base de dados construída a partir de \emph{workflows} reais.

A técnica de recomendação proposta neste projeto é genérica o suficiente para permitir sua aplicação em diferentes contextos, apesar de ter sido projetada para tratar as especificidades da recomendação de atividades em \emph{workflows} científicos.

Os resultados deste mestrado podem auxiliar na redução da redundância de atividades, aumentando a reutilização de atividades existentes e, por conseguinte, reduzindo a necessidade da construção de novas atividades (redundantes) e o tempo necessário para se encontrar a atividade desejada. Segundo \citeonline{CohenBoulakia2014}, uma das principais razões para a falta de reuso de \emph{workflows} e atividades é a grande complexidade destes na área de bioinformática. Segundo o autor \(98,1\%\) dos \emph{workflows} contêm atividades redundantes ou são redundantes em relação a outros \emph{workflows}.

\section{Objetivos} \label{SEC_OBJETIVOS}
O objetivo desse mestrado é criar uma técnica de recomendação de atividades com maior precisão do que as propostas pela literatura. Para tal, será elaborado um sistema híbrido de recomendação de atividades que usará Frequência, Entrada e Saída de atividades, e Ontologia de domínio (FESO).

A hipótese elaborada é: ``Ao acrescentar informação semântica, sobre os dados, para um sistema de recomendação será observada uma melhoria nas métricas \(S@k\) e \(MRR\)''.

Como objetivos secundários, será organizada uma base de dados relacional para armazenar as atividades, os \emph{workflows} e as anotações. Será estudado o domínio de bioinformática para elaborar uma ontologia de domínio, usada para anotar os \emph{workflows} e as atividades. Por fim, será proposta uma nova modelagem do problema de recomendação para solucioná-lo como um problema de classificação ou regressão.

Esta dissertação está organizada com a seguinte estrutura: no capítulo~\ref{CAP_CONCEITOS_FUNDAMENTAIS} são apresentados conceitos básicos deste trabalho, o capítulo~\ref{CAP_CORRELATOS} apresenta a revisão da literatura correlata. O capítulo~\ref{CAP_SOLUCAOPROPOSTA} descreve a solução baseada em ontologias e frequência, o capítulo~\ref{CAP_COMPARACAO} compara as soluções da literatura correlata com as propostas neste mestrado. Por fim, o capítulo~\ref{CAP_CONCLUSOES} conclui o trabalho e apresenta possíveis projetos futuros.