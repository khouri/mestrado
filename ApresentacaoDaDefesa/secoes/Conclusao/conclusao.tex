\section{Considerações finais}



  \begin{frame}
  	\begin{block}{Principais contribui\c{c}\~oes}
  		%Este trabalho teve como objetivo principal especificar uma t\'ecnica para recomendar atividades em \emph{workflows} cient\'ificos. Este objetivo foi alcan\c{c}ado, pois as t\'ecnicas propostas apresentaram resultados superiores aos resultados da literatura. Al\'em deste objetivo prim\'ario foram obtidas as seguintes contribui\c{c}\~oes:
  		\begin{enumerate}
  			\item Uma revis\~ao sistem\'atica sobre a \'area de recomenda\c{c}\~ao de atividades em \emph{workflows} cient\'ificos a qual poder\'a ser a base para trabalhos futuros.
  			\item Foi constru\'ida uma base de dados relacional de \emph{workflows} cient\'ificos com suas respectivas atividades. Esta base ser\'a disponibilizada na \'integra para uso de outros trabalhos.
  			\item Foram implementadas diferentes t\'ecnicas da literatura correlata e foram comparados os resultados da recomenda\c{c}\~ao dessas t\'ecnicas com os resultados da solu\c{c}\~ao proposta.
  			\item At\'e o momento esta pesquisa de mestrado colaborou com a publica\c{c}\~ao de dois artigos cient\'ficos.
  		\end{enumerate}
  		
  	\end{block}
  \end{frame}
	
	\begin{frame}
		\begin{block}{Considerações finais}
			Ao comparar todas as t\'ecnicas, foram constatados determinados aspectos do conjunto de dados, como o fato das atividades n\~ao serem independentes; o problema n\~ao ser linearmente separ\'avel; e que t\'ecnicas de agrupamento n\~ao se mostraram adequadas para solucionar este problema. 
			
			Com exce\c{c}\~ao do SVM, regressores apresentaram solu\c{c}\~oes mais precisas do que classificadores, al\'em disso, adicionar informa\c{c}\~ao nos sistemas de recomenda\c{c}\~ao melhorou a precis\~ao destes.
		\end{block}
	\end{frame}

 \begin{frame}
 	\begin{block}{Trabalhos futuros}
 		%No decorrer deste projeto foram identificadas algumas oportunidades de continuidade e evolu\c{c}\~ao do mesmo, s\~ao elas:
 		\begin{enumerate}
 			\item Usar outros classificadores compostos na recomenda\c{c}\~ao de atividades;
 			\item Criar novas estrat\'egias de recomenda\c{c}\~ao baseadas em redes sociais de pesquisadores ou seus grupos de pesquisa;
 			\item Obter informa\c{c}\~ao sobre proveni\^encia de \emph{workflows} e adicionar esta aos sistemas de recomenda\c{c}\~ao;
 		
		\end{enumerate}		
 	\end{block}
 \end{frame}
 
 
 \begin{frame}
 	\begin{block}{Trabalhos futuros}
 		%No decorrer deste projeto foram identificadas algumas oportunidades de continuidade e evolu\c{c}\~ao do mesmo, s\~ao elas:
 		\begin{enumerate}
 			\item Usar atividades de outros SGWC e/ou de outras \'areas de pesquisa (al\'em da bioinform\'atica);
 			\item Estudar a rela\c{c}\~ao entre a distribui\c{c}\~ao dos dados de entrada (atividade), sua esparsidade e a rela\c{c}\~ao que ambas possuem com o aumento ou redu\c{c}\~ao da precis\~ao das recomenda\c{c}\~oes;
 			\item Utilizar t\'ecnicas de redu\c{c}\~ao de dimensionalidade para o conjunto de dados de entrada;
 			\item Adaptar o classificador SVM para considerar ontologias durante a maximiza\c{c}\~ao da margem \'otima.
 		\end{enumerate}		
 	\end{block}
 \end{frame}